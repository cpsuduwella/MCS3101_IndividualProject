%% ---------------------------------------------------------------
%% This is a template .tex file for M.Sc. theses or reports in
%% Computer Science, Software Engineering or Language Technology
%% in the School of Computer Science at Reykjavík University.
%%
%% It builds on the style RUCSMSc.sty and assumes the 
%% presence of the RU logo in the appropriate format (eps/pdf)
%% 
%% Comments and questions can be sent to 
%% the Graduate Study Council (cs-grad@ru.is)
%%
%% Author: Björn Þór Jónsson (bjorn@ru.is)
%% ---------------------------------------------------------------

%% METHOD:
%% 1) Copy the following files to your working area: RUCSMSc.sty, RU-logo.EPS, RU-logo.PDF
%% 2) Copy this file to your working area
%% 3) Modify this file to fit your needs (please follow all comments below in the text)

%% NOTES:
%% - Assumes several styles are present, including fancyhdr, textpos, ms

%% The following three lines must be as they are
\documentclass[12pt,openright]{report}
\usepackage{RUCSMSc}
\usepackage[T1]{fontenc}
\usepackage[utf8]{inputenc} % added by KK (ShareLaTeX team)

%% Include other packages and your own definitions here

%% If you choose to use apacite (recommended), uncommend this line
%% and the two bibliography lines at the end of the template
% \usepackage{apacite}

\begin{document}

%% Before final submission, the document may be tagged as a draft/proof
%% These commands must be commented out in the final preparation
%% Only use one of these at a time
%\DRAFT{}
%\PROOF{}

%% This template works both for Computer Science and Software Engineering 
%% One or the other must be used
\CS{}
%\SE{}
%\LT{}

%% This template works both for theses and project reports
%% One or the other must be used
%\THESIS{}
\REPORT{}

%% There are two versions: PAPER and ONLINE
%% - The PAPER version produces additional signature pages
%% - The ONLINE version skips the signature pages
%% One of the following must be uncommented, the other commented out
%% Note that final printing must be done in a particular format;
%% information will be available on the web
\PAPER{}
%\ONLINE{}

%% ---------------------------------------------------------------
%% The following commands define the meta-data for the thesis

%% Date in english and icelandic
%% NOTE: THIS IS THE DATE OF THE SUPERVISOR'S SIGNATURE!!!!!!
\MScwhen{Month Year}
\MScdags{Mánuður Ár}

%% Title in english and icelandic
%% Note that title length is limited to what can fit in three lines in the inside page
%% Also note that the capitalization of the text on the front page is a potential source of latex errors
%% as it may not deal well with math, international letters, and other latex constructs
\MSctitle{English Title}
\MScheiti{Íslenskt heiti}

%% If the title must be formatted specifically for the front page or internal pages
%% (typically via line-breaks using the \newline command) then the following commands must be used 
%% This one for the online front page:
\MSctitleF{English \newline Title}
%% This one for the paper-only front page (should not exceed 12 cm in two lines):
%\MSctitleP{English\\Title}
%% These two for the internal pages
%\MSctitleI{English\\Title}
%\MScheitiI{Íslenskt\\heiti}

%% Author name (should be the same in any language:)
\MScauthor{Student name}
%% If the name must be formatted specifically for the signature page
%% (typically via line-breaks) then the following command must be used 
%\MScauthorS{Student\\Name}

%% Give the numbers of members of the thesis committee
%% Note that the style sheet currently handles a maximum of 5 members!
%% If there are more members, consult with the graduate study council...
\MSccmembers{3}

%% List the entire committee.  Each member has a name (degree should be omitted, unless it is not PhD),
%% plus a second argument with title and affiliation.
%% Note that you are responsible for the consistency with \MSccmembers
%% Also note that the MSccommitteeS and MSccommitteeC denote 
%% supervisor and co-supervisor, respectively.
%% Supervisor(s) must appear first
\MSccommitteeS{Name}{Title}{Affiliation}
%\MSccommitteeC{Name}{Title}{Affiliation}
\MSccommittee{Name}{Title}{Affiliation}
\MSccommittee{Name}{Title}{Affiliation}

%% Abstract in english and icelandic
%% It is considered good form to limit the abstract to a single page in each language
\MScabstract{English abstract.}  
\MScutdrattur{Íslenskur útdráttur.}

%% The following two commands define the left and right headers
%% Only use them if the defaults do not look good
%% - This runs on the LEFT hand side of all internal pages
%%   LEFT is authorname
%\MScauthorhead{Type in a shorter version here}
%% - This runs on the RIGHT hand side of pages
%%   RIGHT is title, abbreviated if it is long
%%   Use the main language of the report
%\MSctitlehead{Shorter title, possibly with \dots}

%% This command makes all the front materials!
%% It must produce 6 pages for ONLINE, 8 pages for PAPER

\makeMSc
\pagestyle{plain}

%% ---------------------------------------------------------------
%% Next is a sequence of 6 elements, 3 optional, 3 mandatory
%% There should be no intermediate blank pages!

%% Dedication is optional, comment out if it is absent
\MScdeds{Dedication.} 

%% Acknowledgements are optional, comment out if they are absent
%% Note that it is important to acknowledge any funding that helped in the work
%% They should, however, be here rather than as a chapter at the end
\MScacks{}

Thanks\dots

%% List of publications is optional.  Should contain a comprehensive list 
%% of all publications in which material in the thesis has appeared,
%% preferably with references to sections as appropriate.
%% This is also a good place to state contribution of student and
%% contribution of others to the work represented in the thesis.
\MScpubs{}

Part of the material in this thesis was published \dots

%% TOC, list of figures and list of tables are required
%% (in the unlikely event that one is empty, it should be commented out)
\MSctableofcontents
\MSclistoffigures
\MSclistoftables

%% The list of abbreviations is an example of a special list
%% If used, the comment is provided by the author in regular text
%\MSclistofabbrev{}

%% Other lists may be added, such as lists of algorithms, theorems, etc.
%% If you do, be sure to add them to the table of contents
%% You can consult the definition of \MSclistofabbrev for consistency

%% This command prepares for the actual text, e.g. by 
%% redefining page numbers and footnote numbers
\startMSc

%% ---------------------------------------------------------------
%% From this point on, it is standard Latex, except the very end.
%% This is a "report"-based template, so the top-level heading is \chapter{}

\chapter{Introduction}

\cite{beyer99when}

\chapter{Background}
A. Visible Light Communication
Visible Light Communication acronym as VLC is a novel communication method which uses LEDs' ability to switch into different intensity levels at a fast rate. It is a short range optical wireless communication using visible light spectrum (Fig 01)from 380 to 750 nm [4] . VLC uses LED luminaries for high speed data transfer. LED adaptation has continuously increased and it is expected 75% of total usage by 2030 and this rapid increase in LED usage provides a unique opportunity for communication [4] . LED’s ability to switch into different intensity levels at a very fast rate can be used to transfer data at a high speed without being detected by the human eye [5]–[7]. The idea is to encode the data and send through emitting light and using a photo detector detect at the other end as modulated signals and decode them. Therefore LED is dual purpose, one way as a light source and in the other way as a data communication method. In other terms, it can be used for illumination as well as communication. According to [4] recent research on VLC has shown a very high data rates up to 100Mbps in IEEE 802.15.7 standard and several Gbps in research.

Fig.1: Visible Light Spectrum
It was shown in [8]that flickering can cause serious detrimental physiological changes in humans. For this reason, it is necessary to have changes in the light intensity at a rate faster than a human eye can perceive. IEEE 802.15.7 standard [9] suggests that flickering (or change in light intensity) should be faster than 200 Hz to avoid any harmful effect. That means high data rate will be provided by any VLC system. 
Communication through visible light is important due to many reasons [4] . Firstly, mobile data traffic has increased exponentially in the last two decades and it has proved the fact that RF spectrum is scared to meet ever increasing demand. Compared to that the visible light spectrum is completely untapped for communication and it includes terahertz of unused free bandwidth. Secondly, due to its high frequency, it cannot penetrate through most of the objects and walls. This characteristic allows one to create small cells of LED transmitters with no inter-cell interference issues beyond the walls and partitions. The inability of signals to penetrate through the walls provides an inherent wireless communication security. Thirdly it allows us to use the existing lighting infrastructure for communication as well. Therefore VLC systems can be deployed with less cost and effort. The above reasons motivate us to use VLC for building location-based wireless communication protocol.
In any VLC system, there are two main parts involved, one is the transmitter and the other one is the receiver. LED luminaire is the transmitter of any VLC system. The most important design aspect of a VLC system is that it should not affect the illumination, which is the primary purpose of the luminaire, due to the communication usage. There are two types of receivers; photodetector and image sensor [4]. The image sensor can allow any mobile device with a camera to receive visible light communication. However, this can provide very limited throughput (few Kbps) 
due to its low sampling rate. However, stand-alone photodetectors have a significantly higher throughputs (hundreds of Mbps) In this research, we have used the receiver to be the photodetector in the initial prototype design and the target in future is to replace it with an image sensor, in other words by the camera of the laptop or the mobile device.

B. Wireless Communication and Connection Establishment
Wireless communication is widespread due to its advantages over wired connections [10]. When connecting to a wireless network it is required to select the Access Point (AP) first using the Service Set Identifier (SSID) and if it is secured, a dialog prompts for authentication. Connecting to an AP is a 3 step process. Three steps involved are Discovery, Authentication and Association [11]. 
During the discovery process, the device needs to be connected to the AP listen for beacon frames broadcasted in regular intervals by the AP. When a user tries to connect to the AP, the device sends an authentication request to the AP. 
The Institute of Electrical and Electronics Engineers, Inc. (IEEE) 802.11 standard defines two link-level types of authentication: Open System and Shared Key [12]. Open system authentication consists of two communications. First, an authentication request is sent from the device. Then, an authentication response from the AP/router with a success or failure message will receive. With shared key authentication, a shared key or passphrase is manually set on both the mobile device and the AP/router. Several types of shared key authentication such as WEP, WPA, and WPA2 are available. Only those wireless clients who have the shared key can connect. If there are no passwords set for AP it will be automatically connected same as open Wi-Fi connectivity. But for a password protected Wi-Fi, the AP replies to the authentication request with a challenge in form of text to the device. At this point, we need to provide the password. Then the device encrypts the challenge text sent by the AP with the password and sends back to AP. If the correct password has entered, then the decrypted response will match the initial challenge sent to your device by the AP earlier and then the association stage is initiated with the AP telling the machine that the authentication was successful.
Once authenticated machine will send an association request to AP and once the association acceptance message is received only the device can start transferring data. In the association process, AP and the device get into certain agreements such as the network model, security parameters (either WEP, WPA or WAP2), encryption method (TKIP, CCMP, AES) and channel frequency.
C. Data Security in Wireless Networks
To detect a wireless network all we need is a wireless-equipped device. There is no way to selectively hide the presence of a wireless network from strangers, but prevention of unauthorized people from connecting can be done, and thus can protect the data traveling across the network. Scrambling the data and controlling access to the network can be done by turning on a wireless network's encryption feature. Mentioned below are the most widely used security protocols in wireless networks to provide security and privacy.
1) Wired Equivalent Privacy (WEP)
WEP is not recommended for a secure WLAN. Static client keys for access control made WEP cryptographically weak. The main security risk is the hackers capturing the encrypted form of an authentication response frame, using widely available software applications and using the information to crack WEP encryption. 
2) Wi-Fi Protected Access (WPA)
WPA complies with the wireless security standard and strongly increases the level of data protection and access control (authentication) for a wireless network. WPA enforces IEEE 802.1X authentication and key-exchange and only works with dynamic encryption keys. A common pre-shared key (PSK) must be manually configured on both the client and AP/router.
3) Wi-Fi Protected Access 2 (WPA2)
WPA2 is a security enhancement to WPA. Users must ensure the fact that the mobile device and AP/router are configured using the same WPA version and pre-shared key (PSK).
Key distribution is an important issue in wireless networks. To secure communication between two nodes, a shared cryptographic key between the two nodes must be established. Random key pre-distribution systems provide an efficient approach to the key establishment in such networks that guarantee security against passive attackers.
D. How 802.1x authentication works
The architecture of 802.1x protocol has three main components known as supplicant, access point and authentication server such as Remote Authentication Dial-In User Service (RADIUS). The authentication process begins when the end user attempts to connect to the WLAN. The authenticator or the AP acts as a proxy for the end user passing authentication information to and from the authentication server. The client may send an Extensible Authentication Protocol (EAP) start message. The access point sends an EAP-request identity message. The client's EAP-response packet with the client's identity is "proxied" to the authentication server by the authenticator. The authentication server challenges the client to prove themselves and may send its credentials to prove itself to the client. The client checks the server's credentials and then sends its credentials to the server to prove itself. The authentication server accepts or rejects the client's request for a connection. If the end user was accepted, the AP changes the virtual port with the end user to an authorized state allowing full network access to that end user. At log-off, the client virtual port is changed back to the unauthorized state.
E. Remote Authentication Dial-In User Service (RADIUS) Protocol
Remote authentication dial-in user service or RADIUS is an authentication system that has been used to secure networks. A wireless RADIUS server uses a protocol called 802.1X, which governs the sequence of authentication-related messages that go between the user’s device, the wireless access point (AP), and the RADIUS server. When a user wants to connect to a Wi-Fi network with RADIUS authentication, the device establishes a communication with the AP, and requests access to the network. The AP passes the request to the RADIUS server, which returns a credential request back to the user via the AP. The user provides the proper user name and password, which the RADIUS server checks against the authentication directory. If the credentials are correct, the RADIUS server informs the AP to allow the user access to the network. 
When a user authenticates an SSID using 802.1X, that individual session is encrypted uniquely between the user and the access point. This means that another user connected to the same SSID cannot sniff the traffic and acquire information because they will have a different encryption key for their connection. With a Pre-Shared Key (PSK) network, every device connected to the access point is on a "shared encryption". If you need to de-auth a particular user or device, having RADIUS makes this much easier because you disconnect a single user or device without having to change the key for everyone or allow that potential security risk of that user re-joining the network with the known access key. This special feature has used in the proposed VLC based authentication protocol where keys are dynamically expiring and issuing new keys to ensure the location-based connectivity.
Common home-use Wi-Fi networks may not need a RADIUS server because they "secure" the network with one single network key, the "WPA/WPA2 Pre-Shared Key" (PSK). That key which is same for every user, is often guessable, and can't be revoked for one user.  When a network is sniffed, an attacker can perform offline attacks to guess the key. To provide location constraints, it is mandatory to refresh the key which is assigned to a particular location time to time and allows the user to get the key through a VLC enabled LED placed inside the room.

\chapter{Methods}


\MScpart{Part Name}
\chapter{Experiments}

\chapter{Conclusions}


%% If you choose to use apacite (recommended), uncommend these lines
%% and the package line at the top of the template, otherwise select your bibstyle
%\bibliographystyle{apacite}
%\bibliographystyle{plain}
%\bibliography{main}

%% ---------------------------------------------------------------
%% Following are some relevant options

%% If the dissertation has "parts", the following command may be used
%% It may then, furthermore, be followed by introductory text
%% Note that this should be very rare in MSc theses
% \MScpart{Part Name}
% Introductory text.

%% If appendices are required, uncomment the following line
%% and include the appendices in separate files
%\appendix

%% ---------------------------------------------------------------
%% Produce one blank page (inside the cover) and 
%% one page with the address of the department
\end{document}

%% ---------------------------------------------------------------





















