\section{INTRODUCTION}
Technology, in nowadays plays a major role in each corner of the world and expects the continuous growth. Any person, any business, any service is now interacted with technology, somehow, either fully or up to some extent. The beauty of the technology is that it always changes with new researches and  findings all around the world. People do their researches to test new concepts and introduce new findings to the world to make life easier. With the  introduction of new technologies, market has become really competitive and opens the door to the entire world. People have realized that it is not easy to survive their businesses without having technology. Not only the business world, human life style has also become really closed with the technology and some life styles have been entirely changed due to new technologies. With the broad development of technology, organizations tend to acquire new technologies to the business process, which becomes a new business itself. Even most critical businesses such as financial, industrial also acquire new technologies because of the business competition and for their effectiveness. In order to keep customers, business has to provide the best competitive service to the customer which becomes really pain without new technologies. No matter what kind of business it is, technology is everywhere. 

\paragraph{}
Technology runs our lives these days. Smart phones, tablets and computers \verb|-| we really can't seem to function without them. In a very short amount of time, technology has exploded in the market and now, many people cannot imagine a life without it.  All technologies are born out of purpose. For example, search engines were created to sort through the massive amounts of data online. With the introduction of each new technology, the existing technologies were replaced and the new technologies were able to create something better than what was previously used. And on and on it goes. It has become an integral part of our daily life. We are almost surrounded by technology and we always use technology everywhere and all the time. Computer and internet technologies have changed sectors like medicine, tourism, education, entertainment etc... Technology has touched every aspect of life, making it easier, better and different. It has affected the human life and has changed our living.\cite{technology}

\paragraph{}
Most of the businesses depend on the computer technology as it gives a helping hand to develop and maintain the business. Technology has important effects on business operations. No matter the size of the enterprise, technology has both tangible and intangible benefits that will help to make money and customers. Technological infrastructure affects the culture efficiency and relationships of a business.

\paragraph{Why is technology important in businesses?}
\begin{enumerate}
	\item Cost effectiveness
	\item Communication with customers
	\item Efficiency in operations
	\item Easy in transactions
	\item Accuracy
	\item Accessability
	\item Managability
	\item Data privacy ( security )
	\item Proper data management
	\item High capacity of storage of data
	\item Efficient interactions between the dynamic team within the business
\end{enumerate}

\paragraph{}
With the invention of internet technology, it has become really popular over the world and within online businesses,Electronic commerce(e-Commerce), Electronic retailing (e-Tailling), online shopping, online payment systems,internet banking and many more new concepts have been introduced to the business world and they have become really popular in a short time period. Then after the internet has become a key enabler of the business, even small companies run their businesses with the aid of internet. 

\paragraph{}
Couple of years back, before the introduction of Wi-Fi technology, wired connection (networking) was required to use internet. Each node of the network needs to be connected physically to the network either via hub or switch to connect to the internet. Therefore it required the use of network cables for each node to connect to the network. It is pretty straight forward to connect each node to the network directly, for the purpose of manageability and easy of troubleshooting. But when it comes to large businesses, number of users have increased and become difficult to manage cable systems for the demand. And troubleshooting has become difficult. While with these problems, laptops and hand-held mobile devices were popular and widely used. People needed portability while they are working in their offices which could not be provided by cabling systems. And also people those who lack of knowledge about information technology, experienced uncomfortable because of internet protocol addressing system as well as some network topology. As a solution for this problem, Wi-Fi is introduced.

\paragraph{}
Wi-Fi is a technology for wireless local area network that uses electromagnetic waves (radio waves) to provide wireless high speed internet and network connection\cite{wifi_d}, with devices based on the IEEE 802.11 standards without the use of any cables or wires. It contains Radio signals, Antenna and Router. Like mobile phones, a Wi-Fi network makes the use of radio waves to transmit information across a network. The computer should include a wireless adapter that will translate the data which was sent into a radio signal. This same signal will be transmitted, via an antenna, to a decoder known as the router. Once decoded, the data will be sent to the Internet through a wired connection.  As the wireless network works as a two-way traffic, the data received from the internet will also pass through the router to be coded into a radio signal that will be received by the computer's wireless adapter. \cite{wifi_1}

\paragraph{}
Wi-Fi provides networking facility together with portability and easy in troubleshooting. No need of cabling, no need of wall sockets, no need of wiring everywhere, Wi-Fi technology provides a manageable, smart solution without hazel of much wires and user can easily connect to the network via Wi-Fi access point. Because of these advantages, Wi-Fi has become popular in office premises, open public areas, almost everywhere. Since usability and scalability of Wi-Fi, people tend to use Wi-Fi instead of using wired network connections. Organizations moved quickly to Wi-Fi network connections, even home environment has become a small wireless network unit.This motivates researchers to go for more new concepts such as "Internet of Things" (IOT), "Bring Your Own Device" (BYOD),etc... 

\begin{center}
	\begin{table}[H]			
		\begin{tabu}  { | X[c] | X[c] | }	
			\hline
			Wi-Fi Advantages& Wi-Fi Disadvantages \\
			\hline
			\begin{itemize}
				\item Mobility
				\item Communicates without physical wire connections which reduce the cost
				\item Simple infrastructure
				\item Setup, configuration and maintenance are easy than cabling process
				\item Increased Collaboration(Reduces delays and increases productivity)
			\end{itemize} & 
			\begin{itemize}
				\item Wi-Fi generates radiation which could effect to human health
				\item Signal strength of Wi-Fi depends on physical obstacles, environmental conditions and hardware components which are uncontrollable
				\item Because of the mobility, there is a security risk
			\end{itemize}\\
			\hline
		\end{tabu}
		\caption[Wi-Fi advantages and disadvantages]{Wi-Fi advantages and disadvantages}
		\label{table:aimPt_isol}
	\end{table}
\end{center}


\paragraph{}
With the advancement of technology, more inherent hidden problems are also being exposed. For troubleshoot those, special knowledgeable professionals are required. Easy of accessibility to the network makes life easier not only for authorized users, but also for unauthorized users.Advantages such as easy  of accessibility, portability of Wi-Fi, unauthorized people also tend to connect to the network easily. Those who were succeeded, could do lot of things to the systems and as a subsequence of it, this becomes a really big challenge to the business. Due to risk of personal information and business information leakage, all interactive parities to the system must pay additional attention on the security mechanism of the system and Wi-Fi connection. 

\paragraph{}
Because of the risk related with the Wi-Fi access, new authentication protocols have been introduced for authentication and some of them are outdated and some of them are still in practice to minimize the risk.

\subsection{Wired Equivalent Privacy (WEP)}
WEP\cite{wep} is the first encryption algorithm for the IEEE 801.11b standard, to prevent hackers from snooping on wireless data, when it was transmitted between clients and access point. WEP uses "RC4 Stream Cipher" for encryption. It uses 104 bit size encryption key and the key must be manually entered and updated by an administrator. Reason for the use of RC4 cipher is, that it does not require a powerful CPU to process key. The key consists of 24 bit initialization vector(IV) and due to the small size of IV, it increases the likelihood of being reused and make it easier of being cracked. Because of this vulnarability, using WEP is a risky choice in wireless security, as it can be compromised through WEP attack\cite{wep_attack}

\subsection{Wi-Fi Protected Access (WPA)} 
WPA\cite{wpa_def_2} has been introduced with the improvements of WEP's encryption for a better 802.11i wireless security. The "Temporal Key Integrity Protocol" (TKIP) has also been introduced with WPA, which dynamically generates 128-bit key for each packet. This protocol includes "Message Integrity Check" (MIC) which prevents a packet being altered by an attacker and resending. This has been replaced by the "Cyclic Redundancy Check" (CRC), which is in WEP standards. WPA pre-shared keys are still vulnerable even WPA protocol is more secured than WEP\cite{wpa_vul}.

\subsection{Wi-Fi Protected Access Version2 (WPA2)}
WPA2\cite{wpa_def_2} was introduced to replace WPA to overcome vulnerabilities which were found. WPA2 is also 802.11i standards and it provides AES-based(Advanced Encryption Standard) encryption mode with strong security. But this protocol is still vulnerable to man-in-the-middle attack.

\subsection{Wi-Fi Protected Access Version3 (WPA3)}
WPA3\cite{wpa_def_2} is the latest, which was announced in January 2018, as a replacement of WPA2. This new protocol uses 192-bit length key for encryption and it delivers robust protections for the passwords which are short in complexities and simplify the process of configuration security for devices which have limited display or no display. Protocol provides user privacy in an open network through individual data encryption.\cite{wpa_v3}

\paragraph{}
\begin{center}
	\begin{table}[H]			
	\begin{tabu}  { | X[c] | X[c] | X[c] | }	
	\hline
	Wired Equivalent Privacy (WEP)& Wi-Fi Protected Access (WPA) & Wi-Fi Protected Access Version2 (WPA2) \\
	\hline
	\begin{itemize}
		\item 24 bit initialization key
		\item Uses static encryption keys
		\item Less processing power required
		\item Master key has to be used directly
		\item Can easily exploit vulnerabilities
	\end{itemize} & 
	\begin{itemize}
		\item 48 bit initialization key
		\item Uses unique encryption key
		\item Considerable processing power required
		\item Master key is not using directly
		\item Not that easily exploit vulnerabilities
	\end{itemize} &
	\begin{itemize}
		\item 48 bit initialization key
		\item Uses unique encryption key
		\item More processing power required
		\item Master key is not using directly
		\item Hard to exploit vulnerabilities
	\end{itemize}\\
		\hline
	\end{tabu}
\caption[Wi-Fi security protocol comparison]{Comparison of Wi-Fi security protocols}
\label{table:aimPt_isol}
	\end{table}
\end{center}


\paragraph{}
Since authentication and authorization are key functionalities in a system, specially in security point of view, security of the systems is a really critical and  important basic requirement not only in Sri Lanka, but also in all around the world. Any organization or business, which deals with data should concern about the security of their own systems as well as any third party system which they are dealing with. Not only such entities, most of the individuals who interact with any system, specially concern about their personally identifiable information(PII), privacy and confidentiality. 

\paragraph{}
Therefore, system designers design their systems to provide the best security as possible, with the following basic security characteristics.
	\begin{center}
		\begin{itemize}
			\item Confidentiality
			%	\subitem confidentility description
			\item Integrity
				%				\subitem confidentility description
			\item Availability
					%		\subitem confidentility description
		\end{itemize}
	\end{center}

\subsection{Internet of Things (IoT)}
System of interrelated computing devices,machines, objects, animals or people that are provided with unique identifiers(IP address) and having ability to transfer data over a network without human interaction.\cite{iot}

\paragraph{}
According to the introduction of Internet of Things, a "thing" could be an electronic device, human, machine or even an animal with a biochip transponder or anything. A thing is treated as an individual entity, called object. Each object should have unique identifier called IP address to identify the object uniquely. But Internet Protocol Address Version4 (IPV4), addresses have been almost completely utilized with computers. Scientists introduced a new version of IP address called Internet Protocol Address Version6 (IPV6) to facilitate more devices. Each object is connected to the internet through Wi-Fi. This makes the whole world get connected to each other and makes a complete, one huge network including computers, electronic devices, machines, animals and almost everything. 

\paragraph{}
Because of IoT, if somebody could manage to connect to one of the systems, he/she could be able to connect to more systems, even electric bulb which is in our room and gain control of it. This becomes a really critical situation and to control that risk, each and every system that has been connected to internet should have commonly agreed minimum acceptable level of security to protect not only their own assets, but also any other inter-connected asset which belongs to other country, organization or an individual person.Therefore some individual organizations introduced commonly acceptable level of security countermeasures, later called standards. Some of them became rules and regulations of some countries, regions. However these standards are globally accepted and those who are unable to meet those standards are not allowed to communicate directly with other systems. Therefore these standards have become critical non functional requirements when designing and developing systems and the system should be complianced with those standards in order to communicate with others.

\subsection{Bring Your Own Device (BYOD)}
Another interesting concept, specially introduced for organizations. Many information technology departments struggle to keep up-to-date with rapid technology changes and company employees tend to use their own devices to access company system and company data due to huge business competition. This is an advantage from the company point of view, that company does not require to provide laptops or devices for the employees those who need to access company system for their job. Company can save considerable amount of money without buying laptops and can save maintenance cost. As a part of this concept, BYOD encourages employees to use their own device to get connected to the company network and company data.

\paragraph{}
But like in anything, BYOD also has a darker side. If not properly understood and regulated, this could lead for serious security vulnerabilities and put the organization at a risk. 

\section{PROBLEM}
With the introduction of new technologies and new concepts related to Wi-Fi, more security vulnerabilities have been exposed and risk of data security has been increased. Specially IoT and BYOD concepts deliver more advantages to the organizations while increasing risk. Scientists have introduced countermeasures for the identified vulnerabilities, but still the systems are being attacked by hackers. With the technology, attackers are also getting smart and try to find new ways to exploit vulnerabilities and this game will never be end. Since the available security mechanisms are not enough and being attacked, there is a requirement of a new mechanism of security for Wi-Fi access points.

\paragraph{}
Since Wi-Fi access point can provide its service for several meter radius circle (basically around 20 meters from the center) and also Wi-Fi does not need physical connection to the access point, it could be easily misuse by unauthorized persons without even entering to the office premises.

\paragraph{}
When designing new systems or introducing security functionality to the existing systems, it could be difficult or sometimes can be failed due to some reasons.

\begin{itemize}
	\item It is difficult to introduce new security functionality to the existing systems due to incompatibility of infrastructure and/ or hardware (existing hardware may not support to new functionality due to incompatibility)
	\item New security functionality such as two way authentication or bio-metric authentication requires additional hardware devices which cost high
	\item Sometimes organizations do not have basic level of security and the design of system may not compatible to accommodate basic security functionalities	
	\item Some organizations such as banks  must have industrial information security standards that they should compliance with
	\item Even with really strong standard security mechanisms, still hackers have successfully gained access to the systems due to poor implementation or incomplete implementation	
\end{itemize}

\section{MOTIVATION}
Since most of the organizations and Companies use Wi-Fi, rather than using wired network connections to gain access to the systems, it is really helpful and really worth if there is a smart solution to address those particular problems with low cost together with less infrastructure changes.
As the new concepts BYOD (Bring Your Own Device) and IoT (Internet Of  Things) are being used all over the world, the suggested solution can be easily applied to the systems with low cost. Suggested solution will provide an additional security layer to the system with minimum system changes.

\section{PROPOSED SYSTEM}
LOCATION BASED AUTHENTICATION USING WI-FI SIGNAL STRENGTH provides an additional security layer to the system to protect information assets of the organization by authenticating and authorizing users according to the valid user name, password together with the location where user tries to access to the system.
\subsection{AIM}
The aim of this research project is to build a mechanism to authenticate Wi-Fi users by their user name, password and access location with respect to Wi-Fi signal strength.

\subsection{OBJECTIVES}
\begin{itemize}
	\item Build a mechanism to profile location, based on Wi-Fi signal strength
	\item Create training database according to the location profile
	\item Setup Remote Authentication Dial-In User Service(RADIUS) server to authenticate Wi-Fi users
	\item Create a back-end administration panel for the administrators to handle Wi-Fi users credentials 
	\item Develop a Android mobile application to authenticate users with RADIUS server by providing user name, password and location information with respect to the Wi-Fi signal strength of current location
\end{itemize}

\section{SCOPE}
This proposed solution is designed for pre-defined location of the organization or home and cannot be applied to anywhere without proper pre-analysis of Wi-Fi signal strength. 
\paragraph{}
By using the proposed solution, any defined user who has Android mobile device together with  Android application that specially designed to connect to the system, can be able to connect to the access point, via additional security layer.The Android application keeps the selected three(03) Wi-Fi access points data inside the device, to store access point information and communicate with the RADIUS server, to authenticate user when the user tries to connect to the access point.
\paragraph{}
Users are allowed to connect to the access point only via specially developed Android application with valid user name and password.

\section{LIMITATIONS OF THE RESEARCH PROJECT}
Like in any other concept, LOCATION BASED AUTHENTICATION USING WI-FI SIGNAL STRENGTH also has limitations.

\begin{itemize}
	\item Concept is only applicable for the locations where there is pre-analyzed Wi-Fi information
	\item Minimum of three Wi-Fi access points are required
	\item All three Wi-Fi access points should be permanently fixed 
	\item All three Wi-Fi access points should be actively in use
	\item If any Wi-Fi access point is replaced or hardware is changed, Wi-Fi fingerprint database has to be built again from the beginning before using it
	\item Since Wi-Fi signal strength depends on lot of environmental and physical conditions, the area where this concept is going to be used, should have a controlled environmental conditions as much as possible
\end{itemize}

\section{CONCLUSION OF THE RESEARCH PROJECT}
The test results of the research approach is expected to pass more than \verb|90%|. But the actual results deviated from the expected result. The actual pass rate is less than \verb|90%| and therefore this shows that there could be situations where user is genuine and authorized, but because of the failure of authentication due to poor Wi-Fi signal strength values, user is unable to connect to the access point. Therefore it is recommended to use this system together with some other technique such as New Field Communication (NFC) enabled authentication mechanism as a backup solution. 

\section{HOW THIS REPORT IS ORGANIZED}
\subsection{CHAPTER 2 LITERATURE SURVEY AND REQUIREMENT ANALYSIS}
This chapter describes about the techniques which have been used to gather requirement and the way that those techniques are used. Similar systems that studied to capture technical and functional information are also described within the chapter 2.

\subsection{CHAPTER 3 METHODOLOGY}
This chapter describes the methodology which has been used for the suggested approach in detail.

\subsection{CHAPTER 4 DESIGN}
The design chapter describes about the designing and commonly used software engineering techniques which have been used to design the system. Also describes the detailed design of the proposed system.

\subsection{CHAPTER 5 IMPLEMENTATION}
The Implementation chapter describes about the implementation which has been already done to test the proposed system. Since the proposed approach has several modules, the implementation has been carried out module wise. The special points of each module of the proposed solution  have discussed here.

\subsection{CHAPTER 6 EVALUATION}
In order to assure the quality and functionality of the proposed approach, testing phase is most important. The purpose of having a testing is to make sure that the proposed solution works as expected.
This chapter describes about the testing which has been carried out with the proposed solution.

\subsection{CHAPTER 7 CONCLUSION}
This chapter describes about the results of proposed approach and further modifications that can be done with this solution.

%\subsection{CHAPTER 8 REFERENCES}
%All the references that are used to develop this system are included under this chapter