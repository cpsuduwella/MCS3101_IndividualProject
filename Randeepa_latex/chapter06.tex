\section{OVERVIEW}
Technology has become one of the key factors in business and it plays a critical role in most of the organizations. Introduction of Wi-Fi technology has terminated the wired networking problems and has introduced a mobility networking approach. Along with these advantages of Wi-Fi technology, there are hidden drawbacks like in other technologies. In order to mitigate or reduce those drawbacks, scientists and professionals are still trying for new methods. Wi-Fi technology, is used in many places to carry out really critical functional operations in the business or organization, so as hackers tend to discover vulnerabilities and manage to access systems by exploiting it. 
And sometimes the authorized people manage to gain access to the business critical systems, without entering to the premises or specific area that they are not allowed to. 

\paragraph{}
This research project introduces a new system to authenticate Wi-Fi users by using user name, password as usual and in addition to that, user location. This location factor is with respect to pre-defined Wi-Fi access point signal strengths(Wi-Fi RSSI values).

\paragraph{}
Project objectives were,
\begin{itemize}
	\item Build a mechanism to profile location based on Wi-Fi signal strength
		\subitem To profile location based on Wi-Fi RSSI values, the hybrid technique which is a combination of Wi-Fi trilateration technique and location fingerprinting technique is used.
	
	\item Create training database according to the location profile
		\subitem Location was divided into $1m^2$ equal sizes of each and gathered Wi-Fi RSSI values for each location from fixed three(03) Wi-Fi access points, seperately. To collect Wi-Fi RSSI data, developed android mobile application has been used. By using training database, maximum and minimum Wi-Fi RSSI values of each location per each Wi-Fi access point has been derived.
	
	\item Setup Remote Authentication Dial-In User Service(RADIUS) server to authenticate Wi-Fi users
		\subitem FreeRADIUS server that use RADIUS protocol to authenticate Wi-Fi users has been configured on top of Ubuntu 14.04 host server.
	
	\item Create back end administration panel for the administrators to handle Wi-Fi users credentials 
		\subitem Back end administration panel was designed and developed to manage Wi-Fi users for the system administrators.
	
	\item Develop Android application to authenticate users with RADIUS server by providing user name, password and location information with respect to the Wi-Fi signal strength of current location
		\subitem WPA2Enterprise API integrated android mobile application has been developed to use as a connection tool to Wi-Fi access point. This mobile application is capable of discovering Wi-Fi access points' RSSI values that have been pre-defined and passed those information to FreeRADIUS server by agreed format to authenticate users.
\end{itemize}

The research project was carried out to check whether the user location according to the Wi-Fi signal strength can be used to authenticate users together with user name and password. As per the results, location based authentication using Wi-Fi signal strength can be used for Wi-Fi user authentication process together with some other backup authentication method. The suggesting backup authentication mechanism is facilitated by  using Near Field Communication(NFC) technology with android device and NFC tag which is fixed in a particular location.

\section{FUTURE IMPROVEMENTS}

\begin{itemize}
	\item Since the research project results slightly deviated from the expected results, there is a need of more readings for the location fingerprint training database in several environmental conditions.
	
	\item If the authentication process fails with Wi-Fi RSSI value, there should be an alternative way to get access to the system. The suggested way  introduces NFC enabled authentication mechanism together with location based authentication mechanism.
	
	\item At the moment, the location of the user is only taken for initial authentication. After the user successfully connected to the access point, it does not check user location until the user connects again. Keep monitoring of user's location also has to be introduced to the suggested approach.
	
	\item For the moment, the authentication process completely depends on pre-defined Wi-Fi RSSI database. The module has to be changed to populate training database automatically and response according to that. 
	
	\item Currently, android mobile application and server communicate via plain text JSON object. This should be encrypted by using AES symmetric encryption algorithm.
\end{itemize}
